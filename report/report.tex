\documentclass{aiaa}
\providecommand{\e}[1]{\ensuremath{\times 10^{#1}}}

\title{ASEM 5519: Aerorobotics \\ Project 1, Phase 1}
\author{Matthew Aitken \and Aaron Buysse \and Paul Guerrie \and Steve McGuire
\and Tevis Nichols \and Will Silva}
\begin{document}
\maketitle
\begin{abstract}
This project presents an implementation of Park, Dest, and How's nonlinear guidance algorithm in an ArduPlane environment. The algorihtm is implemented in Python, while the ArduPlane stock firmware code is utilized. Two levels of software simulation results are presented, a crude simulation using a Matlab environment and a software-in-the-loop (SITL) simulation using the actual autopilot code with a fixed-wing simulation engine backend.
\end{abstract}

\section{Control Algorithm Design}
\subsection{As-Published Algorithm}
As published, the Park et al algorithm uses a lookahead distance to determine an intercept angle. The aircraft is then commanded via turn rate to achieve the desired trajectory.
\subsection{Custom Adjustments}
Park et al do not include a means of providing altitude control. As part of the teams's modifications to the base algorithm, we have implemented a proportional-integral-derivative controller around climb rate.  
\section{MATLAB Simulation Results}
\section{Architecture Block Diagrams}
\subsection{Processing Breakdown}
\subsection{Algorithm Initiation Plan}

\subsection{Lost Communications Plan}
The lost communications plan allows our controller to command an aerodynamic termination in the event that signals from the RC radio are not received. The \textbf{DroneAPI} source was modified to present the \textit{rssi} field to client code, representing the received signal strength indication as measured by the Pixhawk autopilot. When the \textit{rssi} drops below a certain threshold, fly-by-wire 'A' mode is engaged, with a throttle override set to the minimum value. The controller then stops sending any additional commands, but stays on-line to monitor whether communications is re-established. If the RC link is re-established, the code can disengage the aerodynamic termination override and set the autopilot back to 'MANUAL' mode, where the RC radio commands are passed through to the control surface servos.
\subsection{Software Sections}
\section{SITL Simulation Results}
\subsection{Lost Communications}
In simulation, the \textit{rssi} field is set to 0; we were able to verify the correct performance of a proof-of-concept implementation of the lost comm strategy. In simulation, the aircraft glides to the ground. This section would require additional verification with the RC remote and physical system before integrating onto an airframe.
\section{Recommendations}
\end{document}